%-------------------------
% Resume in Latex
% Author : Valentin Mitrofanov
%------------------------

\documentclass[letterpaper,12pt]{article}
\usepackage[utf8]{inputenc}
\usepackage[T1,T2A]{fontenc}
\usepackage[english,russian]{babel}

\usepackage{latexsym}
\usepackage[empty]{fullpage}
\usepackage{titlesec}
\usepackage{marvosym}
\usepackage[usenames,dvipsnames]{color}
\usepackage{verbatim}
\usepackage{enumitem}
\usepackage[hidelinks]{hyperref}
\usepackage{fancyhdr}
\usepackage{tabularx}
\usepackage{xcolor}
\usepackage{fontawesome5}
\usepackage[]{mdframed}

\input{glyphtounicode}

% -------------------- FONT --------------------
\pagestyle{fancy}
\fancyhf{} % clear all header and footer fields
\fancyfoot{}
\renewcommand{\headrulewidth}{0pt}
\renewcommand{\footrulewidth}{0pt}

% Adjust margins
\addtolength{\oddsidemargin}{-0.5in}
\addtolength{\evensidemargin}{-0.5in}
\addtolength{\textwidth}{1in}
\addtolength{\topmargin}{-1in} % Default was -.5in
\addtolength{\textheight}{1.0in}

\urlstyle{same}

\raggedbottom
\raggedright
\setlength{\tabcolsep}{0in}

% Section formatting
\titleformat{\section}{
  \vspace{-5pt}\scshape\raggedright\large
}{}{0em}{}[\color{black}\titlerule \vspace{-5pt}]

% Subsection formatting
\titleformat{\subsection}{
  \vspace{-4pt}\scshape\raggedright\large
}{\hspace{-.15in}}{0em}{}[\color{black}\vspace{-8pt}]

% Ensure that generate pdf is machine readable/ATS parsable
\pdfgentounicode=1

% -------------------- CUSTOM COMMANDS --------------------
\newcommand{\resumeItem}[1]{
  \item\small{
    {#1 \vspace{-2pt}}
  }
}

\newcommand{\resumeSubheading}[4]{
  \vspace{-2pt}\item
    \begin{tabular*}{0.97\textwidth}[t]{l@{\extracolsep{\fill}}r}
      \textbf{#1} & #2 \\
      \textit{\small#3} & \textit{\small #4} \\
    \end{tabular*}\vspace{-7pt}
}

\newcommand{\resumeSubSubheading}[2]{
    \item
    \begin{tabular*}{0.97\textwidth}{l@{\extracolsep{\fill}}r}
      \textit{\small#1} & \textit{\small #2} \\
    \end{tabular*}\vspace{-7pt}
}

\newcommand{\resumeProjectHeading}[2]{
    \item
    \begin{tabular*}{0.97\textwidth}{l@{\extracolsep{\fill}}r}
      \small#1 & #2 \\
    \end{tabular*}\vspace{-7pt}
}

\newcommand{\resumeSubItem}[1]{\resumeItem{#1}\vspace{-4pt}}
\newcommand{\resumeSubHeadingListStart}{\begin{itemize}[leftmargin=0.15in, label={}]}
\newcommand{\resumeSubHeadingListEnd}{\end{itemize}}
\newcommand{\resumeItemListStart}{\begin{itemize}}
\newcommand{\resumeItemListEnd}{\end{itemize}\vspace{-5pt}}

\renewcommand\labelitemii{$\vcenter{\hbox{\tiny$\bullet$}}$}

\setlength{\footskip}{4.08003pt}


\begin{document}

\begin{flushright}
  \vspace{-4pt}
  \color{gray}
  \item
\end{flushright}

% -------------------- HEADING--------------------
\vspace{-7pt}
\begin{tabular*}{\textwidth}{l@{\extracolsep{\fill}}r}

  \textbf{\href{https://github.com/mitrofmep/}{\Huge Валентин Митрофанов}} & \faIcon{github}\href{https://github.com/mitrofmep}{{ github.com/mitrofmep}} $  $\\
  {} & \faIcon{linkedin}\href{https://linkedin.com/in/mitrofmep}{{ linkedin.com/in/mitrofmep}} $  $ \\
 \small{Email} : \href{mailto:mitrofmep@gmail.com}{mitrofmep@gmail.com} & \faIcon{code}\href{https://leetcode.com/mitrofmep/}{{ leetcode.com/mitrofmep}} $  $\\
  \small{Mobile} : +7-993-596-4909 & {} $  $ \\
  \faIcon{telegram}\small{ Telegram} : \href{https://t.me/mitrofmep}{{@mitrofmep}} & {} $  $ \\
  
\end{tabular*}


% -------------------- PROJECTS --------------------
\section{Опыт}
    \resumeSubHeadingListStart
        \resumeProjectHeading
        {\textbf{Сбер} $|$ {Java-разработчик - \emph{Полный день}}}
        {\footnotesize\emph{Санкт-Петербург, Россия} $|$ {10/2023 - наст.вр.}}
        \begin{mdframed}
\footnotesize\textbf{Стек: Java 8, 11, 17 / Spring: Boot, Data, WebFlux, Cloud / Apache Maven, Sonatype Nexus / Hibernate, PostgreSQL, CriteriaAPI, MapStruct, Liquibase / Kafka, GraphQL / JUnit, Mockito / BitBucket, Jenkinks CI/CD, SonarQube, OpenShift, WildFly}
\end{mdframed}
        \resumeItemListStart
            \resumeItem{Разрабатывал бэкенд высоконагруженного приложения для сотрудников банка. Поддерживали существующую платформу - многомодульный maven-проект, по сути "фабрика данных" для нескольких смежных команд. Занимался миграцией легаси монолита на микросервисы. Работали по Agile.}
            \resumeItem{По монолиту: Рефакторинг существующих сервисов, добавлял новые фичи, оптимизировал запутанную бизнес-логику, писал шедулеры, скрипты миграции БД (liquibase).}
            \resumeItem{Занимался сборкой релизов, деплоем и развёртыванием в Jenkins}
            \resumeItem{По миграции: При переносе на микросервисы переписывал бóльшую часть кодовой базы, т.к. значительно декомпозировали монолит и меняли бизнес-логику. Разработал модули асинхронного взаимодействия с другими микросервисами. Прошёл весь цикл миграции с нуля до выхода в прод, участвовал в разработке легко масштабируемой архитектуры. Писал код, который служил базой для миграции других микросервисов. }
            \resumeItem{Разработал универсальные расширяемые модули интеграционного тестирования и хэндлеров ответов.}
            \resumeItem{Упорядочил версионирование артефактов в Nexus, уменьшал тех.долг по SonarQube,  раскатывал новые микросервисы в OpenShift.}
            \resumeItem{Проводил код ревью, дополнял бэклог, участвовал в планировании, груминге, демо \\
            (Jira + BitBucket, Confluence).}
        \resumeItemListEnd
        \resumeProjectHeading
        {\textbf{Яндекс} $|$ {Java-разработчик - \emph{Проектная работа}}}{\footnotesize\emph{Сербия, Белград} $|$ {06/2023 - 10/2023}}
                \begin{mdframed}
\footnotesize\textbf{Стек: Java 11 / Spring Boot, Webflux / Apache Maven / JWT, Resilience4j / JUnit, Mockito, Jacoco / Docker, GitLab, GitLab CI, Runner / Swagger UI (OpenAPI)}
\end{mdframed}
        \resumeItemListStart
            \resumeItem{Работал в команде по созданию REST-приложения для интеграции с существующим сервисом Яндекса. Разработали MVP. Всю архитектуру и процессы выстраивали с нуля.}
            \resumeItem{Внёс ключевые оптимизационные изменения в существующий модуль подбора рекомендаций, координировал их с разработчиками из смежных команд, интегрировал API сторонних сервисов. }
            \resumeItem{Разработал компоненты авторизации, обработки токенов и управления cookies.}
            \resumeItem{Самостоятельно осуществлял настройку и поддержку CI/CD, использовали Gitlab Runner, настраивал весь пайплайн раскатки, проводил код ревью.}
            \resumeItem{Проводил облачное нагрузочное тестирование. Контролировал выдачу подробного API для команд фронта и мобилки.}
        \resumeItemListEnd
\newpage
$ $
        \resumeProjectHeading
        {\textbf{\href{https://futurebim.ru/ru}{FutureBIM}} $|$ {Разработчик - \emph{Полный день}}}
        {\footnotesize\emph{Санкт-Петербург, Россия} $|$ {09/2021 - 05/2023}}
                        \begin{mdframed}
\footnotesize\textbf{Стек: Java, Spring: Boot, MVC, Security; Gradle, Hibernate, PostgreSQL, Thymeleaf, JAXB, Jsoup, Jackson, Apache POI, Docker, GitHub}
\end{mdframed}
        \resumeItemListStart
            \resumeItem{Занимался поддержкой внутреннего продукта компании для менеджмента BIM-проектирования: монолитное web-приложение, работало через шаблонизатор, общалось со сторонним десктопным софтом через XML. Написал много сложной бизнес-логики, связанной с преобразованием данных из разных программ и API онлайн-сервисов, получением и визуализацией статистики.}
            \resumeItem{Разработал модуль интеграции с софтом от Autodesk, которая позволила ускорить ключевой флоу приложения на 30\%. Для этого разбирался в легаси модулях на C\# и LISP. Готовил документацию.}
            \resumeItem{Проводил собеседования, онбординги, помогал с выстраиванием процессов в команде, проводил код ревью, писал скрипты по автоматизации проектирования на Python, Bash}
        \resumeItemListEnd
    \resumeSubHeadingListEnd

% -------------------- EDUCATION --------------------
\section{Образование}
  \resumeSubHeadingListStart
  
    \resumeProjectHeading
      {\textbf{Университет ИТМО}}
      {\footnotesize\emph{Санкт-Петербург, Россия} $|$ {2023}}
        \resumeItemListStart
            \resumeItem{{Аспирантура - Факультет программной инженерии и компьютерной техники}} \\
            \resumeItem{\textit{Тема кандидатской диссертации: Методы визуализации для получения совокупной оценки проектных решений}}
        \resumeItemListEnd
    \resumeProjectHeading
      {\textbf{СПбГАСУ}}
      {\footnotesize\emph{Санкт-Петербург, Россия} $|$ {2016-2022}}
        \resumeItemListStart
            \resumeItem{{Бакалавриат, Магистратура - Инженерное проектирование}}
\resumeItemListEnd

  \resumeSubHeadingListEnd

% -------------------- SKILLS --------------------
\section{Навыки}
        \begin{mdframed}
\footnotesize\textbf{Java 8, 11, 17, Core, OOP, Collections, Multithreading, Stream API / Spring Core, Boot, MVC, Security, Data, WebFlux, AOP / Hibernate ORM, JDBC, PostgreSQL, MySQL, H2, CriteriaAPI, MapStruct, Liquibase / Apache Maven, Gradle,  Sonatype Nexus / JUnit, Mockito, JaCoCo / GitHub, Docker, Gitlab CI/CD, Runner, BitBucket, Jenkins, SonarQube / OpenShift, Kubernetes / Linux, Bash / REST, Swagger UI (OpenAPI) }
\end{mdframed}   
\end{document}
